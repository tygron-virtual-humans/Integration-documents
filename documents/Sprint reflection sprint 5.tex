\documentclass{scrartcl}

\usepackage[utf8]{inputenc} % Unicode support (Umlauts etc.)
\usepackage{hyperref} % Add a link to your document
\usepackage{graphicx} % Add pictures to your document
\usepackage{listings} % Source code formatting and highlighting
\usepackage[top=75px, bottom=75px, left=85px, right=85px]{geometry} % Change page borders
\usepackage{color}
%footer
\usepackage{fancyhdr}
\pagestyle{fancy}
\fancyhead{}
\fancyfoot[R]{Context Project: Gamygdala-Integration}
\renewcommand{\headrulewidth}{0pt}
\renewcommand{\footrulewidth}{0.4pt}

\begin{document}

\title{Context Project: Virtual Human}
\subtitle{Group: Gamygdala-Integration}
\date{\today{}}

\author{
    \begin{tabular}{l r}
      B.L.L. Kreynen\\
      M. Spanoghe\\
      R.A.N. Starre\\
      Yannick Verhoog\\
      Joost Wooning\\
    \end{tabular}
}

\maketitle \thispagestyle{empty} \pagebreak

\section{Sprint reflection sprint 5}

\subsection{Main Problems Encountered}

\emph{Problem 1}. We slightly underestimated how difficult it would be to determine the best way to define the relations between beliefs and goals.\\
\emph{Reaction}. We will discuss this with Koen en Joost to see what they think about this.\\ \\
\emph{Problem 2}. After discussing with Koen en Joost we discovered that the way we implemented the emotion base would best be done differently. \\
\emph{Reaction.} We will implement their suggested method in the next sprint.
\\

\subsection{Adjustments for next sprint plan}
As mentioned above next sprint we will add the task of figuring out the best way to define relations between beliefs and goals and the task of implementing the emotionbase.

\end{document}