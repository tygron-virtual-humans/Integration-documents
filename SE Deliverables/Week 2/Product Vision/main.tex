\documentclass[11pt]{article}
\usepackage[utf8]{inputenc}
\usepackage[usenames, dvipsnames]{color}
\renewcommand{\rmdefault}{phv} % Arial
\renewcommand{\sfdefault}{phv} % Arial

\title{Product Vision}
\author{Namen}
\date{April 2015}

\begin{document}

\maketitle

\section*{Introduction}
In this product vision document a description will be given of the customer and his needs, a comparison with existing products and the time frame and budget of the project. The goal of the project is to make a virtual human that can interact with other humans in a game that simulates urban planning. The product described in this document is one part of that project and consists of the emotion simulation part of the virtual human. Some parts of this document will use the global project while others will talk about the emotion part, which part is being talked about will be clear from context. To start of a description of the target customer will be given. 

\section{Customer}
%Who is going to buy the product? Who is the target customer?
The project was started to answer a question for the company Tygron, thus Tygron is also the target customer. Tygron is a company with an engine which goal is to "support city planning by allowing stakeholders to experiment different scenarios in a realistic 3D environment" (\textcolor{red}{referentie naar Tygron website}). This urban planning tool can be used by different parties involved in urban city planning to prevent conflicts and to increase the return on investment of projects. In the next sections a description of the needs of the customer and the product will be given.

\section{Customer needs}
%Which customer needs will the product address?
Tygron's game is most useful when played with multiple parties involved in urban planning at the same time. Currently it is not possible to replace one of of the involved parties by a virtual human. Virtual humans would be able to simulate one or more of the parties that aren't available/willing to play the game, still allowing facilitate planning for the parties that do want to use the Tygron engine. To simulate realistic behaviour for the virtual humans they need to have some strategy which their decisions are based on and, since humans aren't solely strategic, they should have an emotional component that influences their decisions. These emotions should be based on the events happening in the game. In the next section the crucial aspects of the emotion part of the product will be given.

\section{Crucial product attributes}
%Which product attributes are crucial to satisfy the selected needs, and therefore to the
% success of the product?
Some crucial product attributes:
\begin{itemize}
\item Non-Functional:
\begin{itemize}
\item Needs to be able to perform in real-time.
\item Need to use GOAL and GAMYGDALA.
\item Needs to be finished by the 25th of June.
\end{itemize}
\item Functional:
\begin{itemize}
\item Integrate GAMYGDALA with GOAL.
\item ??
\end{itemize}
\end{itemize}



\section{Comparison with existing products}
% How does the product compare against existing products, both from competitors and the
% same company? What are the product’s unique selling points?
% NOTE: Question 4 requires a literature study and analysis of the existing alternatives to the
% envisioned realization of the target software product. 


\section{Time frame and budget}
%What is the target time frame and budget to develop and launch the product?
The draft and final version of the product planning should be delivered on the 7th and the 15th of May, respectively. 
The draft and final version of the emergent architecture design should be delivered on the 30th of April and the 19th of June, respectively.
The draft and final version of the software should be delivered on the 26th of May and the 19th of June, respectively.
The draft and final version of the final report should be deliverd on the 18th of June and the 24th of June, respectively.

\end{document}
